\documentclass[a4paper, 12pt]{article}
\usepackage[UTF8]{ctex}
\usepackage{algorithm}
\usepackage{minted}
\usepackage{listings}
\usepackage{hyperref}[colorlinks]

\begin{document}
\title{函数}
\author{IlleniumDillon}
\date{\today}
\maketitle
定义 函数使用关键字 def,后跟函数名与括号内的形参列表。
函数语句从下一行开始,并且必须缩进。函数内的第一条语句
是字符串时,该字符串就是文档字符串,也称为 docstring。\par
函数在 执行 时使用函数局部变量符号表,所有函数变量赋值
都存在局部符号表中;引用变量时,首先,在局部符号表里查
找变量,然后,是外层函数局部符号表,再是全局符号表,最
后是内置名称符号表。因此,尽管可以引用全局变量和外层函
数的变量,但最好不要在函数内直接赋值(除非是 global 语
句定义的全局变量,或 nonlocal 语句定义的外层函数变量)。\par
\section{默认值参数}
函数定义时,可以给形参指定默认值,调用函数时,可以不传
递默认值参数,这时,函数使用默认值参数。默认值参数必须
放在非默认值参数后面。\par
注意:默认值参数在函数定义时,只计算一次,即在函数定义
时,计算默认值参数的值,而不是在函数调用时计算默认值参数
的值。\par
默认值只计算一次。默认值为列表、字典或类实例等可变对象时,
会产生与该规则不同的结果。例如,下面的函数会累积后续调用
时传递的参数:\par
\begin{listing}[h!]
\begin{minted}{python}
def f(a, L=[]):
    L.append(a)
    return L
print(f(1))
print(f(2))
print(f(3))
# 输出:[1], [1, 2], [1, 2, 3]
\end{minted}
\end{listing}
不想在后续调用之间共享默认值时,应以如下方式编写函数:\par
\begin{listing}[h!]
\begin{minted}{python}
def f(a, L=None):
    if L is None:
        L = []
    L.append(a)
    return L
\end{minted}
\end{listing}

\section{关键字参数}
函数调用时,可以使用关键字参数,即在调用函数时,指定形参
的名称。关键字参数可以在任意顺序中使用,不需要按照函数
定义的顺序。\par
函数调用时,关键字参数必须跟在位置参数后面。所有传递的关
键字参数都必须匹配一个函数接受的参数。最后一个形参为 
$\ast\ast$name 形式时,接收一个字典,该字典包含与函数中
已定义形参对应之外的所有关键字参数。$\ast\ast$name 
形参可以与 $\ast$name 形参组合使用($\ast$name 必须在 
$\ast\ast$name 前面), $\ast$name 形参接收一个 元组,
该元组包含形参列表之外的位置参数。例如,可以定义下面这样
的函数:\par
\begin{listing}[h!]
\begin{minted}{python}
def f(a, b, *args, **kwargs):
    print(a, b)
    print(args)
    print(kwargs)
f(1, 2, 3, 4, 5, x=6, y=7)
# 输出:1 2
# 输出:(3, 4, 5)
# 输出:{'x': 6, 'y': 7}
\end{minted}
\end{listing}
关键字参数在输出结果中的顺序与调用函数时的顺序一致。\par

\section{特殊参数}
默认情况下,参数可以按位置或显式关键字传递给 Python 函数
。为了让代码易读、高效,最好限制参数的传递方式,这样,
开发者只需查看函数定义,即可确定参数项是仅按位置、按位置
或关键字,还是仅按关键字传递。\par
\begin{listing}[h!]
\begin{minted}{python}
def f(pos1, pos2, /, pos_or_kwd, *, kwd1, kwd2):
#     -----------    ----------     ----------
#       |             |                  |
#       |        Positional or keyword   |
#       |                                - Keyword only
#        -- Positional only
\end{minted}
\end{listing}
函数定义中未使用 $\slash$ 和 $\ast$ 时,参数可以按位置或关键
字传递给函数。\par
特定形参可以标记为 仅限位置。仅限位置 时,形参的顺序很重
要,且这些形参不能用关键字传递。仅限位置形参应放在 $\slash$
前。$\slash$ 用于在逻辑上分割仅限位置形参与其它形参。如果函
数定义中没有 $\slash$,则表示没有仅限位置形参。$\slash$ 后可
以是 位置或关键字 或 仅限关键字 形参。\par
把形参标记为 仅限关键字,表明必须以关键字参数形式传递该形
参,应在参数列表中第一个 仅限关键字 形参前添加 $\ast$。
说明:\par
\begin{itemize}
    \item 使用仅限位置形参,可以让用户无法使用形参名。
    形参名没有实际意义时,强制调用函数的实参顺序时,或
    同时接收位置形参和关键字时,这种方式很有用。
    \item 当形参名有实际意义,且显式名称可以让函数定义更
    易理解时,阻止用户依赖传递实参的位置时,才使用关键字。
    \item 对于 API,使用仅限位置形参,可以防止未来修改形
    参名时造成破坏性的 API 变动。
\end{itemize}

\section{任意实参列表}
有时,函数需要接受任意数量的实参,这时,可以使用 $\ast$name
形参。$\ast$name 形参接收一个元组,在可变数量的实参之前,可
能有若干个普通参数,这些普通参数是位置参数,不能用关键字传
递。$\ast$name 形参可以与 $\ast\ast$name 形参组合使用,但必须
在 $\ast\ast$name 形参之前。$\ast$name形参后面的所有形参都是
关键字参数。\par

\section{解包实参列表}
函数调用要求独立的位置参数,但实参在列表或元组里时,
要执行相反的操作。例如,内置的 range() 函数要求独立的 
start 和 stop 实参。如果这些参数不是独立的,则要在调用函
数时,用 $\ast$ 操作符把实参从列表或元组解包出来:\par
\begin{listing}[h!]
\begin{minted}{python}
args = [3, 6]
print(list(range(*args)))
# 输出:[3, 4, 5]
\end{minted}
\end{listing}
同样,字典可以用 $\ast\ast$ 操作符传递关键字参数。\par

\section{匿名函数}
lambda 关键字用于创建小巧的匿名函数。Lambda 函数可用于
任何需要函数对象的地方。在语法上,匿名函数只能是单个表达
式。在语义上,它只是常规函数定义的语法糖。与嵌套函数定义
一样,lambda 函数可以引用包含作用域中的变量。可以把匿名
函数用作传递的实参。\par

\section{文档字符串}
以下是文档字符串内容和格式的约定。\par
第一行应为对象用途的简短摘要。为保持简洁,不要在这里显式
说明对象名或类型,因为可通过其他方式获取这些信息(除非该
名称碰巧是描述函数操作的动词)。这一行应以大写字母开头,
以句点结尾。\par
文档字符串为多行时,第二行应为空白行,在视觉上将摘要与其
余描述分开。后面的行可包含若干段落,描述对象的调用约定、
副作用等。\par

\section{函数注解}
函数注解是关于用户定义函数的元数据信息。注解可以是任何
表达式。注解存储在函数的 \_\_annotations\_\_ 属性中,是一
个字典,键是参数名,值是注解。\par
形参标注的定义方式是在形参名后加冒号,后面跟一个会被求值
为标注的值的表达式。 返回值标注的定义方式是加组合符号 -$>$ ,
后面跟一个表达式,这样的校注位于形参列表和表示 def 语句结
束的冒号。\par
\end{document}