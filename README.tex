% \documentclass 命令必须出现在LaTeX文档的第一行,
% 它用来指定文档的类型。在这里,我们指定文档的类型为article,
% 这意味着我们正在写一篇文章。在方括号中,我们指定了文档的一些属性,
% 比如纸张的大小和字体的大小。在这里,我们指定了纸张的大小为A4,字体的大小为12磅。
% 这些属性是可选的,你可以根据自己的需要来指定。如果你不指定这些属性,LaTeX会使用默认的属性。
\documentclass[a4paper,12pt]{article}
% 在这里,我们使用usepackage命令来引入一些宏包。
% 宏包是一些预先定义好的命令和环境,可以帮助我们更方便地排版文档。
% 我们使用的第一个宏包是ctex。这个宏包是用来支持中文的。
% 我们在方括号中指定了选项UTF8,这意味着我们使用UTF-8编码来编写文档。
\usepackage[UTF8]{ctex}
% 我们使用的第二个宏包是hyperref。这个宏包是用来生成超链接的。
% 我们在方括号中指定了选项colorlinks,这意味着超链接的文字会显示为彩色。
% 如果你不指定这个选项,超链接的文字会显示为黑色。
\usepackage{hyperref}[colorlinks]
% \begin{document} 和 \end{document} 之间的部分是文档的正文部分。
% \begin 之前的部分是导言区,用来指定文档的类型、属性和引入宏包等。
\begin{document}

    % 在这里,我们使用title命令来指定文章的标题。
    \title{Readme of Python Learning}
    % 在这里,我们使用author命令来指定文章的作者。
    \author{IlleniumDillon}
    % 在这里,我们使用date命令来指定文章的日期。
    \date{\today}
    % 在这里,我们使用maketitle命令来生成标题。
    \maketitle  

    % 如果需要的话,可以将文档分成几个部分,每个部分可以放在一个section中。
    % 在这里,我们使用section命令来生成一个部分。
    % 下列分节命令适用于article文档类型,对于其他文档类型,可能会有不同的分节命令。
    % \section{...}
    % \subsection{...}
    % \subsubsection{...}
    % \paragraph{...}
    % \subparagraph{...}
    \section{这是什么?}
        这是一个Python学习的文档,用来记录Python学习的过程。顺便也可以用来练习LaTeX的使用。
    \section{有什么内容?}
        % \href{}{} 命令用来生成超链接。
        % 第一个参数是链接的地址,第二个参数是链接的文字。
        Python学习的主线是廖雪峰的Python教程, 具体参见\href{https://www.liaoxuefeng.com/wiki/1016959663602400}{Python教程}。
        和Python参考手册,具体参见\href{https://docs.python.org/zh-cn/3/reference/index.html#reference-index}{Python参考手册}。除此之外,还会记录一些Python的基础知识和一些代码实践。\par
        LaTeX的学习主要参考\href{https://oi-wiki.org/tools/latex/}{LaTeX入门}。
    \section{进度}
        % \begin{table} 命令用来生成表格。
        % \centering 命令用来将表格居中。
        \begin{table}
        \centering
            % \begin{tabular} 命令用来生成表格的内容。
            % l 表示内容左对齐,c 表示内容居中,r 表示内容右对齐。
            % | 表示生成竖线。
            % \hline 用来生成横线。
            \begin{tabular}{lc}
                项目                    &   更新时间 \\
                \hline
                构建仓库                &   2024-06-01 \\
                基本语法                &   2024-06-02 \\
                数据结构                &   - \\
                流程控制                &   - \\
            \end{tabular}
        \end{table}

\end{document}