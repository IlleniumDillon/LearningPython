\documentclass[a4paper,12pt]{article}
\usepackage[UTF8]{ctex}
\usepackage{algorithm}
\usepackage{minted}
\usepackage{listings}
\usepackage{hyperref}[colorlinks]

\begin{document}
    \title{基本语法}
    \author{IlleniumDillon}
    \date{\today}
    \maketitle

    \section{文件编码}
        默认情况下,Python 源码文件的编码是 UTF-8。
        这种编码支持世界上大多数语言的字符,可以用
        于字符串字面值、变量、函数名及注释。如果不
        使用默认编码,则要声明文件的编码,文件的第
        一行要写成特殊注释。句法如下:
        \begin{listing}[h]
            \begin{minted}{python}
                # -*- coding: encoding -*-
            \end{minted}
        \end{listing}
        其中 encoding 是文件的编码,常见的编码有
        utf-8、gbk、gb2312 等。具体可见\href{https://docs.python.org/zh-cn/3/library/codecs.html#module-codecs}{Python 文档}。
        具体地,如果文件的编码是 utf-8,则第一行应该写成:
        \begin{listing}[h]
            \begin{minted}{python}
                # -*- coding: utf-8 -*-
            \end{minted}
        \end{listing}
        第一行的规则也有一种例外情况,源码以 
        UnixShell 脚本开始,前两行可以写成:
        \begin{listing}[h]
            \begin{minted}{python}
                #!/usr/bin/env python3
                # -*- coding: utf-8 -*-
            \end{minted}
        \end{listing}
        那么第一行的作用是告诉操作系统,这个脚本
        用 Python3 解释器执行,并指出解释器的路径。
        在类 Unix 系统中,可以直接运行这个脚本,
        但是在 Windows 系统中,这两行会被忽略。
    \section{注释}
        Python 的注释以 \# 开头,直到行尾结束。注释
        可以单独占一行,也可以跟在语句后面。注释
        用于解释代码的功能,提高代码的可读性。注释
        可以是单行注释,也可以是多行注释。单行注
        释以 \# 开头,多行注释以三个单引号或三个双
        引号开始和结束。多行注释可以用于函数的文档
        字符串,也可以用于多行注释。
        \begin{listing}[h]
            \begin{minted}{python}
                # 这是一个单行注释
                print("Hello, World!") # 这也是一个单行注释
                '''
                这是一个多行注释
                这是一个多行注释
                这是一个多行注释
                '''
            \end{minted}
        \end{listing}
    \section{缩进}
        Python 使用缩进来表示代码块,缩进是 Python
        语法的一部分。缩进的空格数是可变的,但是
        同一个代码块的语句必须包含相同的缩进空格
        数。一般情况下,缩进使用 4 个空格,也可
        以使用 2 个空格或者 8 个空格。缩进的空格
        数不是固定的,但是同一个代码块的语句必须
        使用相同的缩进空格数。缩进的空格数不是固
        定的,但是同一个代码块的语句必须使用相同
        的缩进空格数。缩进的空格数不是固定的,但
        是同一个代码块的语句必须使用相同的缩进空
        格数。缩进的空格数不是固定的,但是同一个
        代码块的语句必须使用相同的缩进空格数。
        \begin{listing}[h]
            \begin{minted}{python}
                if True:
                    print("True")
                else:
                    print("False")
            \end{minted}
        \end{listing}
    \section{行尾分号}
        Python 语句不需要使用分号结尾,但是如果
        一行中有多个语句,可以使用分号分隔。分号
        用于分隔同一行的多个语句,但是不推荐在同
        一行中写多个语句。如果一行中有多个语句,
        可以使用分号分隔,但是不推荐这样做。
        \begin{listing}[h]
            \begin{minted}{python}
                print("Hello, World!"); print("Hello, Python!")
            \end{minted}
        \end{listing}
    \section{多行语句}
        Python 语句通常以新行开始,但是可以使用
        反斜杠 \ 来实现多行语句。反斜杠 \ 可以用
        于将一行的语句分成多行显示,但是不推荐使
        用反斜杠 \ 来实现多行语句。如果一行的语句
        太长,可以使用反斜杠 \ 将其分成多行显示。
        \begin{listing}[h]
            \begin{minted}{python}
                total = 1 + 2 + 3 + \
                        4 + 5 + 6 + \
                        7 + 8 + 9
            \end{minted}
        \end{listing}
    \section{引号}
        Python 可以使用单引号、双引号和三引号来
        表示字符串。单引号和双引号的作用是相同的,
        三引号用于表示多行字符串。单引号和双引号
        可以用于表示字符串,三引号用于表示多行字
        符串。单引号和双引号的作用是相同的,三引
        号用于表示多行字符串。单引号和双引号的作
        用是相同的,三引号用于表示多行字符串。
        \begin{listing}[h]
            \begin{minted}{python}
                print('Hello, World!')
                print("Hello, Python!")
                print('''Hello, World!
                Hello, Python!''')
            \end{minted}
        \end{listing}
    
\end{document}