\documentclass[a4paper, 12pt]{article}
\usepackage[UTF8]{ctex}
\usepackage{algorithm}
\usepackage{minted}
\usepackage{listings}
\usepackage{hyperref}[colorlinks]

\begin{document}
\title{模块}
\author{IlleniumDillon}
\date{\today}
\maketitle
退出 Python 解释器后,再次进入时,之前在 Python 
解释器中定义的函数和变量就丢失了。因此,编写较长
程序时,最好用文本编辑器代替解释器,执行文件中的
输入内容,这就是编写 脚本 。随着程序越来越长,为
了方便维护,最好把脚本拆分成多个文件。编写脚本还
一个好处,不同程序调用同一个函数时,不用把函数定
义复制到各个程序。\par
为实现这些需求,Python 把各种定义存入一个文件,在
脚本或解释器的交互式实例中使用。这个文件就是 模块 ;
模块中的定义可以 导入 到其他模块或 主 模块(在顶层
和计算器模式下,执行脚本中可访问的变量集)。\par

\section{模块}
模块包含可执行语句及函数定义。这些语句用于初始化模
块,且仅在 import 语句 第一次 遇到模块名时执行。文
件作为脚本运行时,也会执行这些语句。\par
每个模块都有自己的私有命名空间,它会被用作模块中定
义的所有函数的全局命名空间。 因此,模块作者可以在模
块内使用全局变量而不必担心与用户的全局变量发生意外冲
突。 另一方面,如果你知道要怎么做就可以通过与引用模块
函数一样的标记法 modname.itemname 来访问一个模块的全
局变量。\par
模块可以导入其他模块。 根据惯例可以将所有 import 语句
都放在模块(或者也可以说是脚本)的开头但这并非强制要求。
如果被放置于一个模块的最高层级,则被导入的模块名称会被
添加到该模块的全局命名空间。\par
还有一种 import 语句的变化形式可以将来自某个模块的名称
直接导入到导入方模块的命名空间中。\par
\begin{listing}[h!]
\begin{minted}{python}
from fibo import fib, fib2
\end{minted}
\end{listing}
这条语句不会将所导入的模块的名称引入到局部命名空间中
(因此在本示例中,fibo 将是未定义的名称)。\par
还有一种变化形式可以导入模块中的所有名称:\par
\begin{listing}[h!]
\begin{minted}{python}
from fibo import *
\end{minted}
\end{listing}
这种方式会导入所有不以下划线(\_)开头的名称。大多数
情况下,不要用这个功能,这种方式向解释器导入了一批未
知的名称,可能会覆盖已经定义的名称。\par
模块名后使用 as 时,直接把 as 后的名称与导入模块绑定。\par

\subsection{以脚本方式执行模块}
可以用以下方式运行 Python 模块:
\[
python fibo.py <arguments>
\]
这项操作会在模块的全局命名空间内定义一个名为 \_\_name\_\_ 的变量,
并将其设为 \_\_main\_\_。因此,可以通过检查这个变量来确定模块是
以脚本方式执行还是被导入。如果一个模块被导入,\_\_name\_\_ 的值为模块名,如果模块被执行,
\_\_name\_\_ 的值为 \_\_main\_\_。\par
将下列代码添加到模块的末尾:\par
\begin{listing}[h!]
\begin{minted}{python}
if __name__ == "__main__":
    import sys
    fib(int(sys.argv[1]))
\end{minted}
\end{listing}
这个文件既能被用作脚本,又能被用作一个可供导入的模块,
因为解析命令行参数的那两行代码只有在模块作为“main”文件
执行时才会运行,当这个模块被导入到其它模块时,那两行代码不运行。\par

\subsection{模块搜索路径}
当一个模块名被一个 import 语句使用,Python 会在以下位置搜索:
\begin{enumerate}
    \item 当前目录。
    \item 如果模块不在当前目录,Python 会搜索在 shell 变量 PYTHONPATH 中的目录。
    \item 如果没有找到,Python 会搜索 Python 安装时的默认路径。
\end{enumerate}
sys.path 变量包含当前目录, PYTHONPATH 和由 Python 安装决定的默认目录,
初始化后,Python 程序可以更改 sys.path。\par
\begin{listing}[h!]
\begin{minted}{python}
import sys
sys.path.append('/ufs/guido/lib/python')
\end{minted}
\end{listing}

\subsection{dir()函数}
内置函数 dir() 用于按模块名搜索模块定义的名称。返回一个字符串列表,
包含模块定义的所有类型、变量和函数。\par

\section{包}
导入包时,Python 搜索 sys.path 里的目录,查找包的子目录。
需要有 \_\_init\_\_.py 文件才能让 Python 将包含该文件的
目录当作包来处理,在最简单的情况下,\_\_init\_\_.py 可以
只是一个空文件,但它也可以执行包的初始化代码或设置变量。\par
使用 from\ package\ import\ item 时,item 可以是包的子
模块(或子包),也可以是包中定义的函数、类或变量等其他名
称。import 语句首先测试包中是否定义了 item;如果未在包中
定义,则假定 item 是模块,并尝试加载。如果找不到 item,
则触发 ImportError 异常。\par
相反,使用 import\ item.subitem.subsubitem 句法时,除
最后一项外,每个 item 都必须是包;最后一项可以是模块或包,
但不能是上一项中定义的类、函数或变量。\par
\subsection{从包中导入*}
当使用 from\ package\ import\ * 时,如果包的 \_\_init\_\_.py 文件
定义了一个名为 \_\_all\_\_ 的列表,那么在执行 from\ package\ import\ *
时就会导入这个列表中的所有名称。\par
\subsection{包的相对导入}
包的相对导入中,用句点表示相对导入的基准。一个单独的句点表示
当前目录;两个句点表示父目录。\par
注意,相对导入基于当前模块名。因为主模块名永远是 
"\_\_main\_\_" ,所以如果计划将一个模块用作 Python 应用程序
的主模块,那么该模块内的导入语句必须始终使用绝对导入。\par
\subsection{多个目录中的包}
包还支持一个特殊属性 \_\_path\_\_ 。在包的 \_\_init\_\_.py 
中的代码被执行前,该属性被初始化为一个只含一项的列表,该项
是一个字符串,是 \_\_init\_\_.py 所在目录的名称。可以修
改此变量;这样做会改变在此包中搜索模块和子包的方式。\par
\end{document}